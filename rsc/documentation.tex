\documentclass[a4paper,12pt]{article}
\usepackage[T1]{fontenc}
\usepackage{times}
\usepackage{amsmath}
\usepackage{amsfonts}
\usepackage{graphicx}
\usepackage{amssymb}
\usepackage{listings}
\usepackage{textcomp}
\usepackage{hyperref}
\usepackage{xcolor}

\title{Algorytmy numeryczne - projekt}
\author{Szymon Gosk, Damian Górski, Niuta Godlewska}

% commands

\newcommand{\id}{\noindent}
\newcommand{\el}[2]{\lambda_{(#1, #2)}}
\newcommand{\bl}[1]{\textbf{#1}}
\newcommand{\fa}[1]{\displaystyle\mathop{\forall}_{#1}}

\begin{document}

\maketitle

\newpage

\tableofcontents

\newpage

\section{Wstęp}

Niniejszy dokument stanowi dokumentacje wszystkich części projektu przedmiotu \textit{Algorytmy numeryczne}. Kolejne części projektu będą opisane w kolejnych sekcjach, a kolejne elementy poszczególnych częsci - w podsekcjach. \\

\id
Dokument ten jest również podsumowaniem pracy członków zespołu \textbf{1.1}, w którego skład wchodzą \textbf{Szymon Gosk}, \textbf{Damian Górski} i \textbf{Niuta Godlewska}. Członkowie zespołu wspólnie oświadczają, iż projekt był realizowany przez każdego z nich w równym stopniu zaangażowania. \\

\id
Zgodnie z wymaganiami, kod programów został umieszczony w niniejszym dokumencie, jednak dla przejrzystości i wygody czytającego zaleca się zapoznianie się z tym samym kodem na repozytorium projektu: \textcolor{blue}{\href{https://github.com/Szymon-Gosk/szymon-gosk.github.io}{https://github.com/Szymon-Gosk/szymon-gosk.github.io}} \\

\id
Zaimplementowany projekt można znaleźć pod adresem: \textcolor{blue}{\href{https://szymon-gosk.github.io/}{https://szymon-gosk.github.io/}}

\section{Interpolacja wielomianowa : Przyrost naturalny}

\subsection{Opis}

\textbf{Zadanie}: Program, który oszacuje przyrost naturalnyna świecie w przyszłości. Węzły mają przedstawiać przyrost naturalny zmieniający się wczasie. \\

\id
Metoda numeryczna użyta w tym zadaniu to \textbf{metoda newtona}. Początkowo posiadamy zbiór $n+1$ danych: \\

$\{ (x_0, y_0), (x_1, y_1), ..., (x_n, y_n) \}$ \\

\id
Metoda polega na interpolacji danych w wielomian w postaci: \\

$P(x) = \sum\limits_{k=0}^{n}\left( b_k\prod\limits_{i=0}^{k-1}(x-x_i) \right )$ \\

\id
która pozwala oszacować wartości dla argumentów spoza zbioru danych. \\

\newpage

\id
Współczynniki $b_k$ można uzyskać korzystając z poniższego wzoru: \\

$b_k = \left( \frac{b_k \ - \ \sum\limits_{i=0}^{k-1}\left( b_i\prod\limits_{j=0}^{i-1}(x_k-x_j) \right)}{\prod\limits_{i = 0}^{k-1}(x_k - x_i)} \right)$ \\

\id
\bl{Przykład:} Dany jest następujący zbiór danych: \\

$\{ (2005, 6503), (2010, 6894), (2015, 7256) \}$ \\

\id
Zbiór współczynników dla tego zbioru to: \\

$b_0 = 6503$

$b_1 = 78.2$

$b_2 = -0.58$

\id
przez co uzyskujemy wielomian w postaci: \\

$P(x) = 6503 + 78.2(x - 2005) - 0.58(x - 2005)(x - 2010)$ \\

\id
lub po przekształceniach: \\

$P(x) = -0.58x^2 + 2406.9 x - 2487717$ \\

\id
Korzystając z tego wielomianu można przybliżyć wynik dla dowolnej wartości $x$. Dla $x = 2020$ podstawiamy wartość: \\

$P(2020) = -0.58\cdot (2020)^2 + 2406.9\cdot 2020 - 2487717 = 7589$ \\

\id
Kolejnym krokiem jest metoda ogólna przekształcania wielomianu w postaci Newtona w postać ogólną: \\

$\sum\limits_{k=0}^{n}\left( b_k\prod\limits_{i=0}^{k-1}(x-x_i) \right ) = \sum\limits_{k=0}^{n}\left(a_kx^k\right)$ \\

\id
Obecnie, wielomian ma postać: \\

$b_0 + b_1(x-x_0) + b_2(x-x_0)(x-x_1) + ... + b_n(x-x_0)...(x-x_{n-1})$ \\

\id
Pierwszym krokiem będzie wymnożenie kolejnych iloczynów $(x-x_i)$ dla danego współczynnika $b_k$. W tym celu zdefiniujmy macierz $(n+1)\times (n+1)$: \\

$M=\begin{pmatrix}
\lambda_{(0,0)} & \lambda_{(0,1)} & \cdots & \lambda_{(0,j)} & \cdots & \lambda_{(0,n)}\\ 
\lambda_{(1,0)} & \lambda_{(1,1)} & \cdots & \lambda_{(1,j)} & \cdots & \lambda_{(1,n)}\\ 
\vdots & \vdots & & \vdots & & \vdots\\ 
\lambda_{(i,0)} & \lambda_{(i,1)} & \cdots & \lambda_{(i,j)} & \cdots & \lambda_{(i,n)}\\ 
\vdots & \vdots &  & \vdots &  & \vdots\\ 
\lambda_{(n,0)} & \lambda_{(n,1)} & \cdots & \lambda_{(n,j)} & \cdots & \lambda_{(n,n)}\\ 
\end{pmatrix}$ \\

\id
Zamysł elementów $\el{i}{j}$ jest następujący - zapiszmy wielomian w postaci opisanej wyżej: \\

$b_0 + b_1(x-x_0) + b_2(x-x_0)(x-x_1) + ... + b_n(x-x_0)...(x-x_{n-1}) = b_0\el{0}{0} + b_1(\el{1}{1}x+\el{0}{1}) + ... + b_n(\el{n}{n}x^n + \el{n-1}{1}x^{n-1} + ... + \el{0}{n}x^0)$ \\

\id
Innymi słowy, są to iloczyny kolejnych podwielomianów Newtona $\prod\limits_{i=0}^{k-1}(x-x_i)$. \\

\id
Konsekwencją tego jest fakt, iż: \\

$\fa{j<i}\el{i}{j} = 0$ \\

\id
co pozwala nam zapisać macierz $M$ jako: \\

$M = \begin{pmatrix}
\lambda_{(0,0)} & \el{0}{1} & \cdots & \el{0}{j} & \cdots & \el{0}{n}\\ 
0 & \lambda_{(1,1)} & \cdots & \el{1}{j} & \cdots & \el{1}{n} \\ 
\vdots & \vdots & & \vdots & & \vdots\\ 
0 & 0 & \cdots & \lambda_{(i,j)} & \cdots & \el{i}{n}\\ 
\vdots & \vdots &  & \vdots &  & \vdots\\ 
0 & 0 & \cdots & 0 & \cdots & \lambda_{(n,n)}\\ 
\end{pmatrix}$ \\

\id
Dodatkowo, warto zauważyć, że: \\

$\fa{i=j}\el{i}{j}=1$ \\

\newpage

\id
Do uzyskania postaci ogólnej wielomianu, należy dodać do siebie współczynniki $\lambda$ odpowiadające danym potęgom, pomnożone przez odpowiednie współczynniki $b$. Wykorzystamy do tego układ liniowy: \\

$\begin{pmatrix}
\lambda_{(0,0)} & \el{0}{1} & \cdots & \el{0}{j} & \cdots & \el{0}{n}\\ 
0 & \lambda_{(1,1)} & \cdots & \el{1}{j} & \cdots & \el{1}{n} \\ 
\vdots & \vdots & & \vdots & & \vdots\\ 
0 & 0 & \cdots & \lambda_{(i,j)} & \cdots & \el{i}{n}\\ 
\vdots & \vdots &  & \vdots &  & \vdots\\ 
0 & 0 & \cdots & 0 & \cdots & \lambda_{(n,n)}\\ 
\end{pmatrix}\cdot
\begin{pmatrix}
b_0 \\
b_1 \\
\vdots \\
b_i \\
\vdots \\
b_n
\end{pmatrix}
=
\begin{pmatrix}
a_0 \\
a_1 \\
\vdots \\
a_i \\
\vdots \\
a_n
\end{pmatrix}$ \\

\id
Rozwiązaniem tego układu dla poszczególnych $a_i$ jest: \\

$a_i = \sum\limits_{j=0}^{i}b_j\el{i}{j}$ \\

\id
Przy wyznaczaniu współczynników $\lambda$ posłużymy się zbiorem kombinacji z $i$ po $j$, oznaczonym jako $\mathbb{C}_i^j$: \\

$\el{i}{j} = (-1)^{(i+j)}\cdot \left( \sum\limits_{K \in \mathbb{C}_i^j} \ \prod\limits_{k \in K} \left( x_k \right) \right)$ \\

\id
co daje nam wzór ogólny na $a_i$: \\

$a_i = \sum\limits_{j=0}^{i} \left( (-1)^{(i+j)}\cdot b_j\left( \sum\limits_{K \in \mathbb{C}_j^i} \ \prod\limits_{k \in K} \left( x_k \right) \right) \right)$ \\

\id
Teraz możemy zapisać wzór wielomianu w postaci ogólnej: \\

$P(x) = \sum\limits_{i=0}^{n} \left( x^i\sum\limits_{j=0}^{i} \left( (-1)^{(i+j)}\cdot b_j\left( \sum\limits_{K \in \mathbb{C}_j^i} \ \prod\limits_{k \in K} \left( x_k \right) \right) \right) \right)$ \\

\id
\bl{Przykład:} Załóżmy, że mamy wielomian w postaci: \\

$P(x) = b_0 + b_1(x-x_0) + b_2(x-x_0)(x-x_1) + b_3(x-x_0)(x-x_1)(x-x_2)$ \\

\newpage

\id
Zapisujemy równanie liniowe z uzupełnionymi, wyliczonymi według wzoru współczynnikami $\lambda$: \\

$
\begin{pmatrix}
1 & -x_0 & x_0x_1 & -x_0x_1x_2\\
0 & 1 & -(x_0+x_1) & x_0x_1 + x_0x_2 + x_1x_2 \\
0 & 0 & 1 & -(x_0+x_1+x_2)\\
0 & 0 & 0 & 1 \\
\end{pmatrix}
\begin{pmatrix}
b_0 \\
b_1 \\
b_2 \\
b_3 \\
\end{pmatrix}
=
\begin{pmatrix}
a_0 \\
a_1 \\
a_2 \\
a_3 \\
\end{pmatrix}
$ \\

\id
Z czego otrzymujemy: \\

$a_0 = b_0 - b_1x_0 + b_2x_0x_1 - b_3x_0x_1x_2$ \\

$a_1 = b_1 - b_2(x_0 + x_1) + b_3(x_0x_1 + x_0x_2 + x_1x_2)$ \\

$a_2 = b_2 - b_3(x_0 + x_1 + x_2)$ \\

$a_3 = b_3$ \\

\id
A więc: \\

$P(x) = (b_0 - b_1x_0 + b_2x_0x_1 - b_3x_0x_1x_2) + x(b_1 - b_2(x_0 + x_1) + b_3(x_0x_1 + x_0x_2 + x_1x_2)) + x^2(b_2 - b_3(x_0 + x_1 + x_2)) + x^3b_3$ \\

\subsection{Opis implementacji algorytmu}

\id
Implementacja składowych całościowego algorytmu jest prosta z uwagi na matematyczny zapis obliczeń. Każda suma lub iloczyn, jest równaważny pętli w programie. \\

\id
Pierwszym elementem algorytmu jest wyliczenie współczynników $b$. Program oblicza je zgodnie z podanym wzorem, a do liczenia każdej sumy lub iloczynu wykorzystuje pętlę. \\


\id
Po obliczeniu współczynników $b$, program tworzy \textit{String}, w którym zawiera dane współczynniki, wartości $x_i$ i zwraca \textit{String} zawierający postać Newtona wielomianu. \\

\id
Korzystając z powstałego wielomianu program pobiera argument wpisaną przez użytkownika i wylicza wartośc wielomianu dla tego argumentu.

\subsection{Struktury danych i struktura programu}

\id
Program wykorzystuje proste listy, jeden z podstawowych typów danych w języku \textit{Javascript}. \\

\id
Dodatkowo przeprowadzane są operacje na liczbach - \textit{float} - a także, w celu wyświetlania rezultatów, na danych typu \textit{String}. \\

\id
Struktura skryptu obliczeniowego odpowiada warstwie teoretycznej implementacji - tj. kolejnośc wykowywanych zadań w jednym skrypcie jest taka sama jak w sekcji \textbf{2.1} oraz \textbf{2.2}. \\

\subsection{Program}

\id
Poniżej zostały zaprezentowane przykłady działania programu dla różnych przypadków: \\

\textbf{TODO TODO TODO TODO}

\subsection{IO (wejście-wyjście)}



%%%%%%%%%%%%%%%%%%%%%%%%%%%%%%%%%%%%%%%%%%%%%%%%%%%%%%%%%%%%%%%
%%%%%%%%%%%%%%%%%%%%%%%%%%%%%%%%%%%%%%%%%%%%%%%%%%%%%%%%%%%%%%%
%%%%%%%%%%%%%%%%%%%%%%%%%%%%%%%%%%%%%%%%%%%%%%%%%%%%%%%%%%%%%%%
%%%%%%%%%%%%%%%%%%%%%%%%%%%%%%%%%%%%%%%%%%%%%%%%%%%%%%%%%%%%%%%
%%%%%%%%%%%%%%%%%%%%%%%%%%%%%%%%%%%%%%%%%%%%%%%%%%%%%%%%%%%%%%%

\section{Projekt nr 2}

\section{Projekt nr 3}

\section{Projekt nr 4}

\section{Tekst kodu programów}

\subsection{Interpolacja wielomianowa : Przyrost naturalny}

\begin{lstlisting}
$(document).ready(function() {

    // przycisk 'Zatwierdz'
    $('#submit-button').click(function() {

        // inicjalizacja zmiennych
        let inputs = [];
        let years = [];
        let empty = 0;

        // Zebranie wartosci z inputow
        $('.input-y').each(function () {
            if($(this).val() !== "") {
                years.push($(this).val());
            }

        });

        $('.input-v').each(function () {
            if($(this).val() !== "") {
                inputs.push($(this).val());
            }
        });

        // Sprawdzenie czy sa przynajmniej 3 lata TODO
        if(empty > 2) {

        }

        // zamiana stringosw na inty
        let values = [];
        inputs = inputs.map(x => +x);
        years = years.map(x => +x);

        for (let i = 0; i < inputs.length; i++) {
            let temp = [];
            temp[0] = years[i];
            temp[1] = inputs[i];
            values[i] = temp
        }

        // values[X][0] - X-th year       (x)
        // values[X][1] - X-th value      (y)
        let N = values.length;
        let b = [];

        b[0] = values[0][1];
    
        for(let k = 1; k <= N-1; k++) {

            let sum = values[0][1];
            let denominator = 1;

            for(let i = 1; i <= k-1; i++) {
                let product = 1;
                for(let j = 0; j < i; j++) {
                    product *= (values[k][0] - values[j][0]);
                }
                sum += b[i]*product;
            }

            for(let i = 0; i <= k-1; i++) {
                denominator *= (values[k][0] - values[i][0]);
            }

            b[k] = (values[k][1] - sum)/denominator;
        }

        // Wielomian Newtona

        let polynomial = "P(x) = " + b[0];
        let polynomial_graph = "" + b[0];

        for(let k = 1; k <= N-1; k++) {
            if(b[k] < 0) {
                polynomial = polynomial + " - " + Math.abs(b[k]);
                polynomial_graph = polynomial_graph + " - " + Math.abs(b[k]);
            } else {
                polynomial = polynomial + " + " + b[k];
                polynomial_graph = polynomial_graph + " + " + b[k];
            }
            for(let i = 0; i <= k-1; i++) {
                polynomial = polynomial + "(x - " + values[i][0] + ")";
                polynomial_graph = polynomial_graph + "*(x - " + values[i][0] + ")";
            }
        }

        $("#polynomial").html(polynomial);

        // Wyliczenie

        let P = b[0];
        let x = parseInt($('#year-6').val());

        for (let k = 1; k <= N - 1; k++) {
            let ingredient = b[k];
            for (let i = 0; i <= k - 1; i++) {
                ingredient *= (x - values[i][0]);
                polynomial = polynomial + "(x - " + values[i][0] + ")";
            }
            P += ingredient;
        }

        P = Math.round(P * 100) / 100;

        $('#result').html(P);

        // Wykres

        // TODO

        var board = JXG.JSXGraph.initBoard('jxgbox', {boundingbox:[-5000,5000,5000,-5000], axis:true});

        // Macro function plotter
        function addCurve(board, func, atts) {
            return board.create('functiongraph', [func], atts);
        }

        // Simplified plotting of function
        function plot(func, atts) {
            if (atts==null) {
                return addCurve(board, func, {strokewidth:2});
            } else {
                return addCurve(board, func, atts);
            }
        }

        let func = 'function f(x) { ' +
            '   return ' + polynomial_graph + '; ' +
            '} ' +
            'c = plot(f);';

        // Usage of the macro
        function doIt() {
            console.log(':)');
            eval(func);
        }

        doIt();

    });

});
\end{lstlisting}

\subsection{Projekt nr 2}

\subsection{Projekt nr 3}

\subsection{Projekt nr 4}

\end{document}
